
\vskip -1.5cm

\section{Summary}
\setcounter{theorem}{0}

%\renewcommand{\theenumi}{\alph{enumi}}
%\renewcommand{\labelenumi}{\textnormal{(\theenumi)}$\;\;$}
\renewcommand{\theenumi}{\roman{enumi}}
\renewcommand{\labelenumi}{\textnormal{(\theenumi)}$\;\;$}

          %%%%% ~~~~~~~~~~~~~~~~~~~~ %%%%%

\begin{itemize}
\item
	We used \textbf{\color{red}Sentinel-1} data retrieved on \textbf{\color{red}February 24}, 2020.

	There was one beam mode (i.e., IW),
	two relative orbit numbers (4 and 106),
	two channels (i.e., VH and VV),
	three calendar years (2017, 2018, 2019), and
	six wetland types (i.e., marsh, swamp, water, forest, agricultural, shallow water).

\item
	\textbf{Functional Principal Component Analysis (FPCA) I}
	\begin{itemize}

	\item
		We performed the across-year, across-wetland-type FPCA.
		In other words, for each channel (i.e., VH or VV), its discrete time series
		across all years and wetland types were treated as a single group (of discrete time series)
		on which FPCA was performed.

	\item
		Note that, in order to apply FPCA to a group of discrete time series,
		we need these time series to be defined on a common grid of time points.
		However, the radar satellite measurement time series across years do NOT
		share the same grid of time points. 
		Hence, in order to perform FPCA across years, we
		\textbf{\color{red}must standardize the temporal grid across years} somehow.

	\item
		For each (year, wetland type, channel), we began with 1000 discrete time series,
		each corresponding to a geographical location.
		Each time series (location) with one or more missing values was removed.
		A B-spline approximation was independently fitted through each of the remaining time series
		(i.e. those free of missing values).
		We will call this the \textbf{pre-standardized B-spline approximation}
		of the given original time series.

	\item
		Separately, we determined the a common standardized temporal grid.
		To this end, we first determined the maximal common time interval
		across all years and wetland types, as follows:
		The left end-point of this maximal common time interval is simply the maximum of the beginning time points
		of all the time series across years and wetland types,
		whereas its right end-point is the minimum of the finishing time points of all the time series.
		The standardized temporal grid was then obtained by adding 100 equally spaced
		artificial time points within the common time interval.

	\item
		Next, we replaced each original time series with an artificial time series by evaluating its
		B-spline approximation at the standardized temporal grid points.
		(This is possible since each B-spline approximation is a function
		defined over the maximal common time interval.)
		In what follows, we will call this the
		\textbf{standardized B-spline approximation}
		of the original time series.

	\item
		Lastly, functional principal components were computed for all the
		standardized B-spline approximations, across years and wetland types, as a single group.
		The first two principal component scores were retained for each such time series
		in order to visualize the extent of separation in the resulting two-dimensional space
		of the different wetland types.
		See Appendices \ref{FPCA-scatter-original-IW4}, \ref{FPCA-scatter-original-IW106}.
	\end{itemize}

\item
	\textbf{FPCA II}
	\begin{itemize}
	\item
		We proceeded as in FPCA I, but with each time series replaced
		with its ``scaled'' counterpart.
		(More precisely, the scaled time series is obtained from the original time series
		by subtracting its average of the original time series, followed by
		dividing by the standard deviation of the original time series.)
		See Appendices \ref{FPCA-scatter-scaled-IW4}, \ref{FPCA-scatter-scaled-IW106}
	\end{itemize}

\item
	\textbf{FPCA III}
	\begin{itemize}
	\item
		For each (year, time point, location), we have its (VH,VV) measurements as a point in $\Re^{2}$.
		As (year, time point, location) varies, we obtain a collection of data points in $\Re^{2}$.
		We performed ordinary principal components to this collection of data points,
		and replaced each data point -- (VH,VV) measurements, given (year, time point, location) --
		with its resulting OPC scores.
		We then proceeded as in FPCA I, but with the VH and VV time series replaced
		with their corresponding OPC time series.
		See Appendices \ref{FPCA-scatter-opc-IW4}, \ref{FPCA-scatter-opc-IW106}.
	\end{itemize}

\item
	\textbf{FPCA IV}
	\begin{itemize}
	\item
		We then proceeded as in FPCA III, but with each OPC time series replaced
		with its ``scaled'' counterpart.
		See Appendices \ref{FPCA-scatter-opc-scaled-IW4}, \ref{FPCA-scatter-opc-scaled-IW106}.
	\end{itemize}
\end{itemize}

          %%%%% ~~~~~~~~~~~~~~~~~~~~ %%%%%

%\renewcommand{\theenumi}{\alph{enumi}}
%\renewcommand{\labelenumi}{\textnormal{(\theenumi)}$\;\;$}
\renewcommand{\theenumi}{\roman{enumi}}
\renewcommand{\labelenumi}{\textnormal{(\theenumi)}$\;\;$}

          %%%%% ~~~~~~~~~~~~~~~~~~~~ %%%%%
