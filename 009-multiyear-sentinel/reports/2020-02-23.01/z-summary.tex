
\section{Summary}
\setcounter{theorem}{0}

%\renewcommand{\theenumi}{\alph{enumi}}
%\renewcommand{\labelenumi}{\textnormal{(\theenumi)}$\;\;$}
\renewcommand{\theenumi}{\roman{enumi}}
\renewcommand{\labelenumi}{\textnormal{(\theenumi)}$\;\;$}

          %%%%% ~~~~~~~~~~~~~~~~~~~~ %%%%%

\begin{itemize}
\item
	We used \textbf{\color{red}Sentinel-1} data retrieved on \textbf{\color{red}January 27}, 2020.

	There was one beam mode (i.e., IW), two channels (i.e., VH and VV),
	four calendar years (2016, 2017, 2018, 2019), and
	five wetland types (i.e., marsh, swamp, water, forest, agricultural).

\item
	\textbf{Functional Principal Component Analysis (FPCA) I}
	\begin{itemize}
	\item
		For each (year, wetland type, channel), we began with 1000 discrete time series,
		each corresponding to a geographical location.
		Each time series (location) with one or more missing values was removed.
		A B-spline approximation was independently fitted through each of the remaining time series
		(i.e. those free of missing values).
	\item
		For each (year, channel), functional principal components were computed for the
		corresponding the (about 6000) B-spline approximations.
		The first two principal component scores were retained for each such time series
		in order to visualize the extent of separation in the resulting two-dimensional space
		of the different wetland types. See Appendix \ref{FPCA-scatter-original}.
	\end{itemize}

\item
	\textbf{FPCA II}
	\begin{itemize}
	\item
		We proceeded as in FPCA I, but with each time series replaced
		with its ``scaled'' counterpart.
		(More precisely, the scaled time series is obtained from the original time series
		by subtracting its average of the original time series, followed by
		dividing by the standard deviation of the original time series.)
		See Appendix \ref{FPCA-scatter-scaled}.
	\end{itemize}

\item
	\textbf{FPCA III}
	\begin{itemize}
	\item
		For each (year, time point, location), we have its (VH,VV) measurements as a point in $\Re^{2}$.
		As (year, time point, location) varies, we obtain a collection of data points in $\Re^{2}$.
		We performed ordinary principal components to this collection of data points,
		and replaced each data point -- (VH,VV) measurements, given (year, time point, location) --
		with its resulting OPC scores.
		We then proceeded as in FPCA I, but with the VH and VV time series replaced
		with their corresponding OPC time series.
		See Appendix \ref{FPCA-scatter-opc}.
	\end{itemize}

\item
	\textbf{FPCA IV}
	\begin{itemize}
	\item
		We then proceeded as in FPCA III, but with each OPC time series replaced
		with its ``scaled'' counterpart.
		See Appendix \ref{FPCA-scatter-opc-scaled}.
	\end{itemize}
\end{itemize}

          %%%%% ~~~~~~~~~~~~~~~~~~~~ %%%%%

\vskip 0.5cm
\noindent
\textbf{Comments / Observations}
\begin{itemize}
\item
	The results based on the January 27, 2020 Sentinel-1 data appear promising.
	Marsh, swamp/forest, water, and agricultural lands appear \textbf{well separated},
	\textbf{consistently across years}
	(except for some slight inconsistencies between 2016 and the other years).
	See Appendices \ref{FPCA-scatter-original} and \ref{FPCA-scatter-opc}.
\item
	However, regrettably, \textbf{\color{red}swamp and forest were not well separated}.
\item
	The most effective variable appears to be the first ordinary principal component (computed from VH, VV),
	followed closely by the original channel VV.
	Compare Appendices \ref{FPCA-scatter-opc} and \ref{FPCA-scatter-original}
	against \ref{FPCA-scatter-scaled} and \ref{FPCA-scatter-opc-scaled}.
\item
	One obvious possible explanation for the inconsistencies between 2016 and the other years
	may be the \textbf{\color{red}missing time points} in the 2016 data.
\item
	\textbf{\color{red}Missing measurements}:
	There were 579 combinations of (time point, location) in the 2016 data with missing (VH,VV) measurements. 
	All of the 2016 missing data points pertained to forest, water or agricultural lands.
	There were no missing data for 2017.
	There were three and one missing data points for 2018 and 2019, respectively, all pertaining to agricultural lands.
\end{itemize}

          %%%%% ~~~~~~~~~~~~~~~~~~~~ %%%%%

%\renewcommand{\theenumi}{\alph{enumi}}
%\renewcommand{\labelenumi}{\textnormal{(\theenumi)}$\;\;$}
\renewcommand{\theenumi}{\roman{enumi}}
\renewcommand{\labelenumi}{\textnormal{(\theenumi)}$\;\;$}

          %%%%% ~~~~~~~~~~~~~~~~~~~~ %%%%%
