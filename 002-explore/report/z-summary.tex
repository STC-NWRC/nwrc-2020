
\section{Summary}
\setcounter{theorem}{0}

%\renewcommand{\theenumi}{\alph{enumi}}
%\renewcommand{\labelenumi}{\textnormal{(\theenumi)}$\;\;$}
\renewcommand{\theenumi}{\roman{enumi}}
\renewcommand{\labelenumi}{\textnormal{(\theenumi)}$\;\;$}

          %%%%% ~~~~~~~~~~~~~~~~~~~~ %%%%%

\begin{itemize}
\item
	We used radar satellite data retrieved on January 14, 2020.

	There were two beam/swath positions (FQ5W and FQ17W) and
	four calendar years (2016, 2017, 2018, 2019).
	However, of the $8$ possible
	(beam/swath, year) combinations, the previous ``separation'' procedure
	(based on ordinary and functional principal component analysis) only worked for
	four. The rest failed due to missing data.
\item
	For each beam/swath position, and each calendar year, a transformation
	based on a combination of ordinary, as well as functional principal component analysis,
	was applied to the corresponding $4$-dimensional polarimetric time series
	in order to attempt to separate marshes from swamps.
	\begin{itemize}
	\item
		More precisely, for each beam/swath position, data from 1000 locations of marshes 
		and 1000 locations of swamps were used.
	\item
		Ordinary principal component analysis was applied to this combined data set
		(here considered a finite subset of $\Re^{4}$).
	\item
		For each location and each time point, only the first ordinary principal component score
		was retained for further analysis.
		In other words, in downstream analysis, we worked only with the resulting
		discrete time series of first principal component scores of each location.
	\item
		We next individually scale the time series of first principal component scores
		of each given location, by subtracting the mean and dividing by the standard deviation
		of the time series.
	\item
		Functional principal component analysis was applied to the scaled time series of
		first ordinary principal component scores.
	\item
		A scatter plot of the first and second of the corresponding
		functional principal component scores was generated for each location. 
	\end{itemize}
\item
	However, of the $8$ possible (beam/swath, year) combinations,
	the above procedure only worked for four. The rest failed due to missing data.
\item
	For the four (beam/swath, year) combinations for which the marsh/swamp "separation" procedure
	described above worked, marsh and swamp appeared indeed well separated
	in the $2$-dimensional space spanned by the first two functional principal compoents.
\item
	However, this exercise may need to be re-done, due to the several issues that
	affect the data (retrieved on January 14, 2020).
	\begin{itemize}
	\item
		large number of missing values for some (beam/swath, year) combinations,
	\item
		uneven spacing of measurement time points,
	\item
		some of the 2017 data files contained data from 2018.
	\end{itemize}
\end{itemize}

          %%%%% ~~~~~~~~~~~~~~~~~~~~ %%%%%

\vskip 0.5cm
\noindent
\textbf{Questions}
\begin{itemize}
\item
	Why were the FQ20W data ``cleaner'' (e.g. no missing values, equal gap between consecutive measurements) ?
\end{itemize}

          %%%%% ~~~~~~~~~~~~~~~~~~~~ %%%%%

\vskip 0.5cm
\noindent
\textbf{Comments / Next Steps}
\begin{itemize}
\item
	Perhaps need to re-do this exercise on a ``cleaner'' data set.
\item
	Include more beam/swath combinations.
\item
	Include more types of wetland.
\item
	Perform virtual ``historical production runs'': 
	For each given calendar year, test whether a ``separation model''
	trained on data strictly prior to the given year can effectively and accurately separate
	different types of wetland based on data from the current calendar year.
\end{itemize}

          %%%%% ~~~~~~~~~~~~~~~~~~~~ %%%%%

%\renewcommand{\theenumi}{\alph{enumi}}
%\renewcommand{\labelenumi}{\textnormal{(\theenumi)}$\;\;$}
\renewcommand{\theenumi}{\roman{enumi}}
\renewcommand{\labelenumi}{\textnormal{(\theenumi)}$\;\;$}

          %%%%% ~~~~~~~~~~~~~~~~~~~~ %%%%%
